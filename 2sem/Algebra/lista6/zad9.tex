\documentclass[a4paper]{article}
% Kodowanie latain 2
%\usepackage[latin2]{inputenc}
\usepackage[T1]{fontenc}
% Można też użyć UTF-8
\usepackage[utf8]{inputenc}

% Język
\usepackage[polish]{babel}
% \usepackage[english]{babel}

% Rózne przydatne paczki:
% - znaczki matematyczne
\usepackage{amssymb}
\usepackage{amsmath, amsfonts}
% - wcięcie na początku pierwszego akapitu
\usepackage{indentfirst}
% - komenda \url 
\usepackage{hyperref}
% - dołączanie obrazków
\usepackage{graphics}
% - szersza strona
\usepackage[nofoot,hdivide={2cm,*,2cm},vdivide={2cm,*,2cm}]{geometry}
\usepackage{lmodern}
\usepackage{changepage}
\frenchspacing
% - brak numerów stron
\pagestyle{empty}
\begin{document}
\begin{center}
{\fontsize{20}{20}\selectfont Zadanie 9}
\end{center}

Niech A będzie macierzą kwadratową n x n, rozpatrzmy układ równań postaci:
$$A\overrightarrow{X} = \overrightarrow{B}$$
\begin{adjustwidth}{30pt}{30pt}
\textbf{1.} Gdy $det(A) = 0$ to z własności wyznacznika $rk(A) < n$. Jeśli dla pewnego $i$ $det(A_{x_{i}}) \neq 0 $ to znowu z własności wyznacznika $rk(A_{x_{i}}) = n $. Z tego wynika, że również $rk([A|B]) = n $. Z twierdzenia Kroneckera-Capellego układ równań jest sprzeczny.

$\hspace{420pt}\blacksquare$\\
\textbf{2.} Niech 
$ A = 
\begin{bmatrix}
    0 & 0\\
    0 & 0
  \end{bmatrix}
$,
$ B = 
\begin{bmatrix}
    1\\
    0
  \end{bmatrix}
$ oraz 
$ [A|B] = 
\begin{bmatrix}
    0 & 0 & 1\\
    0 & 0 & 0
  \end{bmatrix}
$, wtedy $det(A) = 0$ oraz dla każdego $i$ $det(A_{x_{i}}) = 0 $. Jednak $rk(A) = 0$, $rk([A|B] = 1)$ zatem z twierdzenia Kroneckera-Capellego układ równań jest sprzeczny.

$\hspace{420pt}\blacksquare$

\end{adjustwidth}
\end{document}
