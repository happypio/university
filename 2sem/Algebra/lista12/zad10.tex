\documentclass[a4paper]{article}
% Kodowanie latain 2
%\usepackage[latin2]{inputenc}
\usepackage[T1]{fontenc}
% Można też użyć UTF-8
\usepackage[utf8]{inputenc}

% Język
\usepackage[polish]{babel}
% \usepackage[english]{babel}

% Rózne przydatne paczki:
% - znaczki matematyczne
\usepackage{amssymb}
\usepackage{amsmath, amsfonts}
% - wcięcie na początku pierwszego akapitu
\usepackage{indentfirst}
% - komenda \url 
\usepackage{hyperref}
% - dołączanie obrazków
\usepackage{graphics}
% - szersza strona
\usepackage[nofoot,hdivide={2cm,*,2cm},vdivide={2cm,*,2cm}]{geometry}
\usepackage{lmodern}
\usepackage{changepage}
\usepackage{tikz}
\frenchspacing
% - brak numerów stron
\pagestyle{empty}
\begin{document}
\begin{center}
{\fontsize{20}{20}\selectfont Zadanie 10}
\end{center}
Rozpatrzmy poszczególne przypadki z zadania:\\
\begin{adjustwidth}{30pt}{30pt}
\textbf{1.} - 4 wierzchołki czerwone i 0 białych. Z prostej obserwacji wynika, że jest tylko 1 rozróżnialny kwadrat (wystarczy zastosować na nim np. identycznośc). Gdy dodamy symetrię odpowiedź się nie zmieni.\\
\textbf{2.} - 3 wierzchołki czerwone i 1 biały.\\
\begin{center}
\begin{tikzpicture}
\draw (0,0) -- (2,0) -- (2,2) -- (0,2) -- (0,0);
\foreach \Point in {(2,0),(2,2),(0,2)}{
    \node at \Point {\textbullet};
    \node [red] at \Point {\textbullet};
}
\node at (0,0) {\textbullet};
\node [white] at (0,0) {\textbullet}; 

\end{tikzpicture}
\end{center}
Zauważmy, że wsytarczy obrócić kwadrat o jeden z kątów - 90\textdegree, 180\textdegree ,270\textdegree aby przenieść biały wierzchołek w dowolny inny. Zatem dla obrotów jak i obrotów z symetrią mamy tylko 1 rozróżnialny kwadrat.\\
\textbf{3.} - 2 wierzchołki czerwone i 2 białe.\\
\begin{center}
\begin{tikzpicture}
\draw (0,0) -- (2,0) -- (2,2) -- (0,2) -- (0,0);
\foreach \Point in {(2,0),(2,2)}{
    \node at \Point {\textbullet};
    \node [red] at \Point {\textbullet};
}
\node at (0,0) {\textbullet};
\node [white] at (0,0) {\textbullet};
\node at (0,2) {\textbullet};
\node [white] at (0,2) {\textbullet};

\draw (4,0) -- (6,0) -- (6,2) -- (4,2) -- (4,0);
\foreach \Point in {(4,0),(6,2)}{
    \node at \Point {\textbullet};
    \node [red] at \Point {\textbullet};
}
\node at (6,0) {\textbullet};
\node [white] at (6,0) {\textbullet};
\node at (4,2) {\textbullet};
\node [white] at (4,2) {\textbullet}; 
\end{tikzpicture}
\end{center}
Intuicyjnie to powinny być jedyne dwa rozróżnialne kwadraty dla obrotu jak i obrotu z symetrią, sprawdźmy to wykorzystując lemat Burnside'a:\\
$$|A| = \frac{1}{|G|} \sum_{g \in G}fix(g) $$
Gdzie $|A|$ to ilość orbit, czyli szukane rozróżnialne kwadraty. Mamy 8 przekształceń:\\
\\ przekształcenie identyczności $e$ - możliwe kolorowanie wierzchołków to ($ (4\cdot 3) \div 2 = 6 $) i każde z nich jest punktem stałym.\\
\\ obrót o 90 \textdegree - 0 punktów stałych \\
\\ obrót o 180 \textdegree - sytuacja, kiedy narożniki naprzeciwko (względem przekątnej) siebie mają ten sam kolor, czyli 2 punkty stałe \\
\\ obrót o 270 \textdegree - 0 punktów stałych \\
\\ symetria wzdłóż jednej z przekątnych - dla jednej przekątnej są 2 punkty stałe, narożniki naprzeciwko (względem przekątnej) siebie mają ten sam kolor \\ 
\\ symetria przez bok - dla jednego boku są 2 punkty stałe, wtedy kiedy narożniki wzdłóż tego samego boku mają ten sam kolor \\
Dla grupy obrotów kwadratu ze złożeniem przekształceń mamy:
$$|A| = \frac{1}{4} \cdot (0 + 6 + 0 + 2) = 2$$
Dla grupy obrotów i symetrii kwadratu ze złożeniem przekształceń mamy:
$$|A| = \frac{1}{8} \cdot (0 + 6 + 0 + 2 + 4 + 4) = 2$$
Zatem w przypadku, gdy 2 wierzchołki są białe istnieją 2 różne kwadraty.\\
\textbf{4.} - 3 wierzchołki białe i 1 czerwony. Analogicznie jak w 2. , 1 rozróżnialny kwadrat.\\
\textbf{5.} - 4 wierzchołki białe i 0 czerwonych. Analogicznie jak w 1. , 1 rozróżnialny kwadrat.
\end{adjustwidth}
\end{document}
