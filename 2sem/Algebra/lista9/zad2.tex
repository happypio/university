\documentclass[a4paper]{article}
% Kodowanie latain 2
%\usepackage[latin2]{inputenc}
\usepackage[T1]{fontenc}
% Można też użyć UTF-8
\usepackage[utf8]{inputenc}

% Język
\usepackage[polish]{babel}
% \usepackage[english]{babel}

% Rózne przydatne paczki:
% - znaczki matematyczne
\usepackage{amssymb}
\usepackage{amsmath, amsfonts}
% - wcięcie na początku pierwszego akapitu
\usepackage{indentfirst}
% - komenda \url 
\usepackage{hyperref}
% - dołączanie obrazków
\usepackage{graphics}
% - szersza strona
\usepackage[nofoot,hdivide={2cm,*,2cm},vdivide={2cm,*,2cm}]{geometry}
\usepackage{lmodern}
\usepackage{changepage}
\frenchspacing
% - brak numerów stron
\pagestyle{empty}
\begin{document}
\begin{center}
{\fontsize{20}{20}\selectfont Zadanie 2}
\end{center}
\begin{adjustwidth}{30pt}{30pt}
\textbf{1.}Weźmy dowolną macierz symetryczną $M$. Z zadania 1 wiemy, że:
$$\langle u , Mv \rangle = \langle M^{T}u,v \rangle$$
$M$ jest symetryczna, zatem $M^{T}u = Mu$ co daje nam:
$$\langle u , Mv \rangle = \langle Mu,v \rangle$$
\textbf{2.} Jeśli $\lambda$ i $\lambda'$ są różnymi wartościami własnymi macierzy symetrycznej M o wektorach własnych $v$ i $v'$ to $Mv = \lambda v$ oraz $Mv' = \lambda v'$. Z (1.) wiemy , że:
$$\langle v' , Mv \rangle = \langle Mv',v \rangle$$
$$\Updownarrow$$
$$\langle v' , \lambda v \rangle = \langle \lambda' v',v \rangle$$
$$\Updownarrow$$
$$\lambda \langle v' , v \rangle = \lambda' \langle v',v \rangle$$
$$\Updownarrow$$
$$\lambda \langle v' , v \rangle - \lambda' \langle v',v \rangle = 0$$
$$\Updownarrow$$
$$(\lambda - \lambda') \langle v' , v \rangle = 0$$
Ponieważ $\lambda \neq \lambda'$ to $\langle v, v' \rangle = 0$, czyli $v$ i $v'$ są prostopadłe.
\end{adjustwidth}
\end{document}
