\documentclass[a4paper]{article}
% Kodowanie latain 2
%\usepackage[latin2]{inputenc}
\usepackage[T1]{fontenc}
% Można też użyć UTF-8
\usepackage[utf8]{inputenc}

% Język
\usepackage[polish]{babel}
% \usepackage[english]{babel}

% Rózne przydatne paczki:
% - znaczki matematyczne
\usepackage{amssymb}
\usepackage{amsmath, amsfonts}
% - wcięcie na początku pierwszego akapitu
\usepackage{indentfirst}
% - komenda \url 
\usepackage{hyperref}
% - dołączanie obrazków
\usepackage{graphics}
% - szersza strona
\usepackage[nofoot,hdivide={2cm,*,2cm},vdivide={2cm,*,2cm}]{geometry}
\usepackage{lmodern}
\usepackage{changepage}
\frenchspacing
% - brak numerów stron
\pagestyle{empty}
\begin{document}
\begin{center}
{\fontsize{20}{20}\selectfont Zadanie 3}
\end{center}

Weźmy dowolną macierz kwadratową $M$, taką że $ \lambda ^2 $ jest wartością własną macierzy $M^2$. Wtedy zbiór wektorów własnych dla $M^2$ to $ker(M^2 - \lambda ^2Id)$. Weźmy dowolny wektor $V \in ker(M^2 - \lambda ^2Id)$:
$$V \in ker(M^2 - \lambda ^2Id)\Leftrightarrow (M^2 - \lambda ^2Id)V = 0 \Leftrightarrow (M - \lambda Id)(M + \lambda Id) V = 0 $$
To oznacza, że $(M + \lambda Id) V = 0 \quad \lor \quad (M - \lambda Id) [(M + \lambda Id) V] = 0$,
ponieważ  $(M + \lambda Id) V$ to wektor, możemy za niego podstawić wektor W i otrzymamy $(M - \lambda Id) W = 0$,  wtedy:\\

\begin{adjustwidth}{30pt}{30pt}
\textbf{1.} Gdy $(M + \lambda Id) V = 0$ to $V \in ker(M - (-\lambda)Id)$ zatem $-\lambda$ jest wartością własną macierzy $M$.\\
\textbf{2.} Gdy $(M - \lambda Id) W = 0$ to $W \in ker(M - (\lambda)Id)$ zatem $\lambda$ jest wartością własną macierzy $M$.\\
\end{adjustwidth}
Zatem jeśli $ \lambda ^2 $ jest wartością własną macierzy $M^2$ to $-\lambda$ lub $\lambda$ jest wartością własną macierzy $M$.

\hspace{460pt}$\blacksquare$
\end{document}
