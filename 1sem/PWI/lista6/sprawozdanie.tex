\documentclass[a4paper]{article}
% Kodowanie latain 2
%\usepackage[latin2]{inputenc}
\usepackage[T1]{fontenc}
% Można też użyć UTF-8
\usepackage[utf8]{inputenc}

% Język
\usepackage[polish]{babel}
% \usepackage[english]{babel}

% Rózne przydatne paczki:
% - znaczki matematyczne
\usepackage{amsmath, amsfonts}
% - wcięcie na początku pierwszego akapitu
\usepackage{indentfirst}
% - komenda \url 
\usepackage{hyperref}
% - dołączanie obrazków
\usepackage{graphics}
% - szersza strona
\usepackage[nofoot,hdivide={2cm,*,2cm},vdivide={2cm,*,2cm}]{geometry}
\frenchspacing
% - brak numerów stron
\pagestyle{empty}
\usepackage{filecontents,listings,graphicx,varwidth}
% dane autora
\author{Piotr Piesiak,961365601}
\title{Przykładowy plik w systemie \LaTeX}
\date{\today}

% początek dokumentu
\begin{document}
\maketitle cos tam
\section{polecenie id}
polecenie id sluzy do wyswietlenia informacji o uzytkowniku i grupie, budowa:
id nazwa-uzytkownika, jesli nie podamy nazwy to domyslnie sprawdzimy informacje o uzytkowniku aktualnie zalogowanym. Dostajemy ifnormacje na temat: do jakiej grupy nalezy uzytkownik, jaki numer majat te grupy.


\section{czas uniksowy}
sposob przedstawienia czasu jako liczba sekund od poczatku 1970 roku czasu UTC.
Nie uwzglednia on sekund przestepnych - tzn. sekund, ktore sie dodaje w roku w celu zsynchronizowania
czasu UTC z czasem słonecznym. (spowodowane jest to zmianami obortu ziemi). W systemie unix czas uniksowy przedstawiany jest jako 32 bitowa liczba sekund, ktore uoplynely od 1 stycznia 1970 roku.
Poniewaz liczbe sekund interpetuje sie jako liczbe ze znakiem, w ktorej wartosci ujemne nie sa wykorzystywane, dostepny przedzial czasu wynosi limit int. W zwiazku z ograniczonym przedzialem,19 stycznia 2038 roku o godzinie 03:14:07 czasu UTC, ilosc sekund wykorzysta cala pojemnosc limitu inta. W nastepnej sekundzie czas uniksowy (przekroczy wartosc zatem zmieni znak na -) zacznie wskazywac date w roku 1901 lub gdy sie wyzeruje, poczatek mierzenia czasu w 1970. Moze to spowodowac bledy w programach korzystajacych z uniksowego czasu.






\section{roznica miedzy kompilowaniem raz, a kompilowaniem dwa razy}
Za pierwszym razem kompilator gdy widzi reflinki zapisuje je w pliku log bo jeszcze nie doszedl do konca pliku i nie wie do czego sie odnosza, za drugim razem gdy juz wie jak wyglada caly plik moze polaczyc reflinki z odnosnikami dzieki czemu reflinki staja sie aktywne



\section{czwarty rozdzial}
awdakjnj








\section{zwierzatko}
\begin{verbatim}
Art by Joan G. Stark
  _      _                        
 : `.--.' ;              _....,_  
 .'      `.      _..--'"'       `-._
:          :_.-'"                  .`.
:  6    6  :                     :  '.;
:          :                      `..';
`: .----. :'                          ;
  `._Y _.'               '           ;
    'U'      .'          `.         ; 
       `:   ;`-..___       `.     .'`.
       _:   :  :    ```"''"'``.    `.  `.
     .'     ;..'            .'       `.'`
    `.......'              `........-'`
\end{verbatim}
\ \\
\hspace*{0,15cm}\_\ \ \ \ \ \ \_\\
/\ \ $\backslash$.-"""-./\ \ $\backslash$\\
$\backslash$ \ \ -\ \ \ -\ \ /\\
\hspace*{0,1cm}|\ \ o\ \ \ o\ \ |\\
\hspace*{0,1cm}$\backslash$\ \ .-'''-.\ /\\
\ \ '-$\backslash$\_Y\_/-'\\
\hspace*{0,6cm}`--`


\end{document}

